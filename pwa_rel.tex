We now describe how PWA relational modeling can be used to enrich
discrete abstractions of black box systems. We first describe the idea
of enriched relations. We then show how the annotated reachability
graph can be interpreted as a PWA model or as a PWA relational model.
Finally, we show that both are instances of a $k$-relational
translation from a reachability graph, which we use in our approach.

\subsection{Enriched Abstraction}

The existential abstraction relation in \chapref{abs} was defined as
follows:
\[
%\exists \x\in\C. \exists \x'\in\C'. \x'=\simulate(\x, \tau) \iff \C\rel{\tau}\C'
    C \areach{t} C' \iff \exists \x \in C.\ \exists \x' \in C'.\ \x \areach{t} \x'
\]
The abstract relation $\areach{t}$ can be enriched by incorporating
the affine relation between $\x$ and $\x'$. For an arbitrary dynamical
system, such a relation can be rarely represented using an exact affine map.
This is due to the presence of non-linear and hybrid behaviors; but,
an affine map can always be estimated with an error.

If the system dynamics are completely specified in the form of a white
box model, we can use first order approximations to find the affine
expressions for the relations. Using a tool like \flowstar, we can
obtain over-approximate affine maps of the form $A\x + [\vb^l,
\vb^h]$, where $[\vb^l,\vb^h]$ denotes the interval of vectors such
that, every element $b_i$ of the vector $\vb$ lies between the
respective scalar interval $b_i\in[b^l_i,b^h_i]$.

For the case of black box systems, such a sound approximation is not
possible. Instead, we rely on a statistical method like simple linear
regression to estimate the affine map $\amap: A\x + \vb$. We then use
cross-validation to estimate the error $\delta$ and generalize
$\amap$ to an interval affine map as before $\amap: A\x + \vb \pm
\delta$.

Finally, we get an abstraction with the below relation
\[
    C \areach{A\x+\vb+\delta} C' \iff \exists \x \in C.\ \exists \x' \in C'.\
    \x' \in A\x + \vb \pm \delta.
\]

We now compute a PWA model for a given black box system by estimating
the dynamics using OLS. We can estimate the affine relations between
every cell, but as discussed in ~\chapref{abs}, it is a futile
approach.  Instead, we use the same heuristic as before:
scatter-and-simulate, to select the cell relations for estimation. We
build the reachability graph as before, but in addition, annotate each
edge with the values of estimated $A$, $\vb$, and $\delta$ for the
respective relation. The reachability graph thus computed, is a
transition system. Using off-the-shelf bounded model checkers, we
reason about the system's safety properties.


\subsection{PWA Model ($0$-relational)}
When $k=0$, the $0$-relational model is a PWA system which only
defines the evolution of states $\x$, and does not specify the
reachable cell. The guards of the guarded affine maps are defined over
cells: $\guard_i(\x) \equiv \x \in C_i$.

We use regression to estimate the dynamics for the outgoing trajectory
segments form a cell $C$. Hence, the data set $\ds$ for the regression
includes all trajectory segments beginning from the same cell
\[
    \ds = \setof{\pi_t | start(\pi_t) \in C}.
\]
This includes trajectory segments ending in different cells.  For $n$
cells, this results in a PWA model (and a transition system), as
shown below.

\begin{equation}
    \pwa = \left\{
        \begin{array}{ll}
            \gm{1} = A_1\vx + \vb_1 + \delta_1 & \vx \in C_1\\
            \gm{2} = A_2\vx + \vb_2 + \delta_2 & \vx \in C_2\\
            \ldots & \ldots\\
            \gm{n} = A_n\vx + \vb_n + \delta_n & \vx \in C_n
        \end{array}
    \right.
\end{equation}

Note that this can be quite imprecise when the cells are big,
containing regions of state-space with complex dynamics. This is true
for both non-linear systems and hybrid dynamical system, where a cell
can contain two or more modes with differing continuous dynamics.

\subsection{PWA Relational Model ($1$-relational)}

To improve the preciseness of learnt dynamics, we include the
reachability relation (as discussed before this section) in the
regression. For every relation $C\areach{t}C'$, the data set $\ds$ is
comprised only of trajectory segments $\pi_t$ which start and end in
the same set of cells.
\[
    \ds = \setof{\pi_t | start(\pi_t) \in C \land end(\pi_t) \in C'}.
\]
The resulting $1$-relational model is at least as precise as the
corresponding $0$-relational model.

\subsection{$k$-relational PWA Model}

Finally, we generalize the PWA relational models to $k$-relational PWA
models. A $k$-relational model is constructed by using $k$ length
\textit{connected} segmented trajectories. A segmented trajectory $\SegTraj$ is
\textit{connected} iff its cost $\cost(\SegTraj) = 0$.

Two $\SegTraj:\tupleof{\pi_{t_1}, \ldots, \pi_{t_k}}$ and
$\SegTraj':\tupleof{\pi'_{t_1}, \ldots, \pi'_{t_k}}$ are
\textit{similar} if their trajectory segments have the same sequence
of cell traversals $\tupleof{C_1, C_2, \ldots, C_k}$, \ie
\[
    \forall i\in\setof{1 \ldots k}.
    start(\pi_{t_i}) \in C_i
    \land start(\pi'_{t_i}) \in C_i
    \land end(\pi_{t_i}) \in C_{i+1}
    \land end(\pi'_{t_i}) \in C_{i+1}
\]

For every cell $C$ in the reachability graph, we construct a data set
$\ds$ by collecting $\pi_{t_i}$, such that they are the first segment of $k$
length \textit{connected} and \textit{similar} segmented trajectories.
%\[
%    \ds = \setof{\pi_{\tau_i} |
%        \pi_{\tau_i}
%        start(\pi_{\tau_{i}}) \in C_i
%        \land
%        end(\pi_{\tau_{i}}) \in C_i}
%\]



