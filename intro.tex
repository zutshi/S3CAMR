
In \chapref{abs}, we analyzed systems by restricting ourselves to
black box semantics. This enabled us to introduce an approach
independent of the system's structure (in practice). By observing the
dynamics only locally and sparsely (due to a coarse abstraction), we
were able to efficiently find abstract counter-examples. However, to
concretize them, we were forced to `take a closer look' or refine the
abstraction. As noted previously, splitting the state-space evenly is
an expensive operation in higher dimensions and we would like to avoid
it. We address this by using learning techniques to model the given
black box system by a PWA relational model.

In this chapter, we propose an approach which uses regression to
quantitatively estimate the discovered relations (witnessed by
trajectory segments) by affine maps. These maps are summaries of
locally observed behaviors, which we incorporate into the sampled
reachability graph as edge annotations. The resulting graph can be
interpreted as a discrete transition system and model checked for
time bounded safety properties. If a violation is discovered, it is
expected to be `close' to a true violation of the system.

\section{Overview}

In this chapter, we further our exploration of trajectory segment
based methods; and propose another approach for searching safety
violations in dynamical systems. Unlike the CEGAR based approach in
\chapref{abs} which directly uses the sampled reachability graph, we
first model quantitatively the local behaviors (represented by edges).
Recall that the edges of the graph represent observed trajectory
segments, between the respective cells. From this, we can only
conclude that a cell is reachable but cannot comment on the
underlying dynamics. If however, we estimate the dynamics of the
witnessed trajectory segments between two cells, we can model the
local dynamics. As we use affine templates to achieve this, it can be
compared to trajectory \textit{linearization}.
% We differentiate it by noting that

The dynamics can be either estimated soundly or only approximated with
error estimates. For the former, we require the white box model of a
system. Using a reach-set estimation tool like
\flowstar~\cite{chen2013flow}, the reachable states of an abstract
state can be soundly computed. As this is not usually the case, in the
rest of the chapter we focus on the case of black box systems.
Recall that such systems are equipped with a simulation function
$\simulate$, but the internal details of the model are opaque to us.
We use simple regression to fit an affine map to the locally
observed behaviors and use cross-validation to estimate the errors.
This results in an interval affine map, approximating a reachable
hyper-rectangle in the state-space.

Observe that the directed reachability graph can be interpreted as a
discrete transitions system. However, an explicit model checker can
only find abstract counter-examples in it. The graph does not have
enough information to enable the direct search of concrete
counter-examples. We remedy this by incorporating the estimated local
dynamics by annotating the graph edges with them. The resulting
transition system is rich enough that an off-the-shelf bounded model
checker can find concrete violations of a safety property, thus doing
away with the expensive refinement process. We now discuss the
background required to present our ideas.
