\documentclass[defaultstyle,11pt]{mythesis}

\usepackage{hyperref}		% PDF hyperreferences??


% packages just included...did not check for duplications or if they
% are actually required!
\RequirePackage{xspace}
\usepackage{float}
\usepackage{rotating}
\usepackage[ruled,vlined,linesnumbered]{algorithm2e}
\usepackage{syntonly}
\usepackage{stmaryrd}
%\usepackage{subfigure}
\usepackage{wrapfig}
\usepackage{import}
\usepackage{enumerate}
\usepackage{amsmath,amssymb,amsfonts}
\pagestyle{plain}
\usepackage{times}
\usepackage{multirow}
\usepackage{multicol}
\usepackage{fancyhdr}
\usepackage{balance}
\usepackage[sort]{cite}
%\usepackage[cmex10,fleqn]{amsmath}
\usepackage{array}
\usepackage{booktabs,siunitx}
\usepackage{url}
% traj_opt
\usepackage{tikz}
\usetikzlibrary{arrows,backgrounds,decorations,decorations.pathmorphing,positioning,fit,automata,shapes,snakes,patterns}
\usepackage{graphicx}
\usepackage{caption}
\usepackage{subcaption}
%\usepackage{subfig}
\usepackage{color}
\usepackage{xcolor,colortbl}
\usepackage{microtype}
\usetikzlibrary{arrows,backgrounds,positioning,fit,automata,shapes,snakes,patterns}
\usetikzlibrary{arrows,backgrounds,decorations,decorations.pathmorphing,positioning,fit,automata,shapes,snakes,patterns}

\usepackage[utf8]{inputenc}
\usepackage{latexsym}
\usepackage{nameref}

% symex
\usepackage{framed}
\usepackage{ragged2e}
\usepackage{varwidth}
\usetikzlibrary{arrows,backgrounds,positioning,fit,automata,shapes,snakes,patterns}
\usetikzlibrary{arrows,backgrounds,decorations,decorations.pathmorphing,positioning,fit,automata,shapes,snakes,patterns}
\usetikzlibrary{shapes.geometric, arrows, positioning, calc, matrix}
% \tikzset{block/.style = {draw, fill=blue!20, rectangle,
%                          minimum height=3em, minimum width=6em},
%         sum/.style = {draw, fill=blue!20, circle, node distance=1cm},
%         input/.style = {coordinate},
%         output/.style = {coordinate},
%         pinstyle/.style = {pin edge={to-,thin,black}}
% }

%\usepackage{siunitx}
\usepackage{listings}

% \makeatletter
% \def\url@mystyle{%
%   %\@ifundefined{selectfont}{\def\UrlFont{\sf}}{\def\UrlFont{\small\ttfamily}}}
%   \@ifundefined{selectfont}{\def\UrlFont{\mathtt}}{\def\UrlFont{}}}
%   %\@ifundefined{selectfont}{\def\UrlFont{\sf}}{\def\UrlFont{\small}}}
% \makeatother
% %\urlstyle{mystyle}
% \urlstyle{rm}


\definecolor{mygreen}{rgb}{0,0.6,0}
\definecolor{mygray}{rgb}{0.5,0.5,0.5}
\definecolor{mymauve}{rgb}{0.58,0,0.82}

%%%%%%%%%%%%   All the preamble material:   %%%%%%%%%%%%

\title{Reachability Analysis of Cyber-Physical Systems using Symbolic-Numeric Techniques}

\author{Aditya Krishna}{Zutshi}

\otherdegrees{B.E., Manipal University, 2007 \\
	      M.S., University Colorado, 2011}

\degree{Doctor of Philosophy}		%  #1 {long descr.}
	{Ph.D., Electrical Engineering}		%  #2 {short descr.}

\dept{Department of}			%  #1 {designation}
{Electrical, Computer, and Energy Engineering}		%  #2 {name}

\advisor{Prof.}				%  #1 {title}
	{Sriram Sankaranarayanan}			%  #2 {name}
        %\reader{}{}		%  3rd person to sign thesis
        \reader{Prof.\;Fabio Somenzi}		%  3rd person to sign thesis
        \readerThree{Prof.\;Bor-Yuh Evan Chang}		%  2nd person to sign thesis
        \readerFour{Dr.\;Jyotirmoy V. Deshmukh}		%  3rd person to sign thesis
        \readerFive{Dr.\;James Kapinski}		%  3rd person to sign thesis

\abstract{  \OnePageChapter	% because it is very short

    In this thesis, we address the problem of reachability analysis in
    cyber-physical systems. These are systems engineered by
    interfacing computational components with the physical world. They
    provide partially or fully automated safety-critical services in
    the form of medical devices, autonomous vehicles, avionics and
    power systems.

    We propose techniques to reason about the reachability of such
    systems, and provide methods for falsifying their safety
    properties. We model the cyber component as a software program and
    the physical component as a hybrid dynamical system. Unlike model
    based analysis, which uses either a purely symbolic or a numerical
    approach, we argue in favor of using a combination of the two. We
    justify this by noting that the software program running on a
    computer is completely specified and has precise semantics. In
    contrast, the model of the physical system is only an
    approximation. Hence, we treat the former as a white box, but
    treat the latter as a black box.

    Using symbolic methods for the cyber components and numerical
    methods for hybrid systems, we carefully capture the complex
    behaviors of software programs and circumvent the difficulty in
    analyzing complex models developed through first principles. To
    combine the two techniques, we use a Counterexample Guided
    Abstraction Refinement (CEGAR) framework. Furthermore, we explore
    learning techniques like regression and piecewise affine modeling
    to estimate and represent black box hybrid dynamical systems for
    the purpose of falsification.

    We use prototype implementations to demonstrate the effectiveness
    of presented ideas. Using non-trivial benchmarks, we compare their
    performance against the state of the art. We also comment on their
    applicability and discuss ideas for further improvement.



    %% OLD
%    In this thesis, we addresses the problem of reachability analyses
%    in cyber-physical systems. These systems are the result of cyber
%    components interacting with the physical world. They provide
%    partially or fully automated safety-critical services in the form
%    of medical devices, autonomous vehicles, avionics and power
%    systems.
%    %but not limited to health care as medical devices, personal and
%    %public transportation as critical safety features in cars,
%    %auto-pilots in airplanes, and, in managing sensitive
%    %infrastructure like power plants.
%
%    We propose techniques to reason about the reachability of such
%    systems, in order to provide methods for falsifying their safety
%    properties. We model the cyber component as a software program and
%    the physical component as a hybrid dynamical system. Unlike model
%    based analysis techniques which use either a purely symbolic or a
%    numerical approach, we argue in favor of using the combination of
%    the two.
%    %We justify this by highlighting the differences in the two models
%    %available in practice.
%    To elaborate, the completely specified software program running on
%    a microprocessor has precise semantics. In contrast, the physical
%    system is only approximately modeled in practice. Hence, we use
%    white-box techniques for the former, but treat the latter as a
%    black-box.  This allows us to carefully capture the complex
%    behaviors of softwares and circumvent the difficulty in analyzing
%    complex models developed through first principles.
%
%    We use symbolic techniques for the cyber components and propose
%    numerical techniques for hybrid systems. We then put forth a
%    Counterexample Guided Abstraction Refinement (CEGAR) framework to
%    combine the two techniques and provide a procedure to falsify
%    safety properties of the system. Furthermore, we explore learning
%    techniques like regression and piecewise affine modeling to
%    estimate and represent relevant symbolic information required to
%    falsify the property.
%
%    We use prototype implementations to demonstrate the effectiveness
%    of presented ideas. We use non-trivial benchmarks to compare their
%    performance against the state of the art. We also comment on their
%    applicability and discuss ideas for further improvement.

}
\dedication[Dedication]{	% NEVER use \OnePageChapter here.
	}

\acknowledgements{	\OnePageChapter	% *MUST* BE ONLY ONE PAGE!
	}

% \IRBprotocol{E927F29.001X}	% optional!

\ToCisShort	% use this only for 1-page Table of Contents

\LoFisShort	% use this only for 1-page Table of Figures
% \emptyLoF	% use this if there is no List of Figures

\LoTisShort	% use this only for 1-page Table of Tables
% \emptyLoT	% use this if there is no List of Tables

\definecolor{mygreen}{rgb}{0,0.6,0}
\definecolor{mygray}{rgb}{0.5,0.5,0.5}
\definecolor{mymauve}{rgb}{0.58,0,0.82}

\newtheorem{example}{Example}[section]
\newtheorem{definition}{Definition}[section]
\newtheorem{assumption}{Assumption}[section]
\newtheorem{lemma}{Lemma}[section]
\newtheorem{proof}{Proof}[section]
%\newtheorem{theorem}{Theorem}[section]

%%%%%%%%%%%%%%%%%%%%%%%%%%%%%%%%%%%%%%%%%%%%%%%%%%%%%%%%%%%%%%%%%
%%%%%%%%%%%%%%%       BEGIN DOCUMENT...         %%%%%%%%%%%%%%%%%
%%%%%%%%%%%%%%%%%%%%%%%%%%%%%%%%%%%%%%%%%%%%%%%%%%%%%%%%%%%%%%%%%

\begin{document}

\input macros.tex
\input generic.macros.tex
\input specific.macros.tex



In \chapref{abs}, we analyzed systems by restricting ourselves to
black box semantics. This enabled us to introduce an approach
independent of the system's structure (in practice). By observing the
dynamics only locally and sparsely (due to a coarse abstraction), we
were able to efficiently find abstract counter-examples. However, to
concretize them, we were forced to `take a closer look' or refine the
abstraction. As noted previously, splitting the state-space evenly is
an expensive operation in higher dimensions and we would like to avoid
it. We address this by using learning techniques to model the given
black box system by a PWA relational model.

In this chapter, we propose an approach which uses regression to
quantitatively estimate the discovered relations (witnessed by
trajectory segments) by affine maps. These maps are summaries of
locally observed behaviors, which we incorporate into the sampled
reachability graph as edge annotations. The resulting graph can be
interpreted as a discrete transition system and model checked for
time bounded safety properties. If a violation is discovered, it is
expected to be `close' to a true violation of the system.

\section{Overview}

In this chapter, we further our exploration of trajectory segment
based methods; and propose another approach for searching safety
violations in dynamical systems. Unlike the CEGAR based approach in
\chapref{abs} which directly uses the sampled reachability graph, we
first model quantitatively the local behaviors (represented by edges).
Recall that the edges of the graph represent observed trajectory
segments, between the respective cells. From this, we can only
conclude that a cell is reachable but cannot comment on the
underlying dynamics. If however, we estimate the dynamics of the
witnessed trajectory segments between two cells, we can model the
local dynamics. As we use affine templates to achieve this, it can be
compared to trajectory \textit{linearization}.
% We differentiate it by noting that

The dynamics can be either estimated soundly or only approximated with
error estimates. For the former, we require the white box model of a
system. Using a reach-set estimation tool like
\flowstar~\cite{chen2013flow}, the reachable states of an abstract
state can be soundly computed. As this is not usually the case, in the
rest of the chapter we focus on the case of black box systems.
Recall that such systems are equipped with a simulation function
$\simulate$, but the internal details of the model are opaque to us.
We use simple regression to fit an affine map to the locally
observed behaviors and use cross-validation to estimate the errors.
This results in an interval affine map, approximating a reachable
hyper-rectangle in the state-space.

Observe that the directed reachability graph can be interpreted as a
discrete transitions system. However, an explicit model checker can
only find abstract counter-examples in it. The graph does not have
enough information to enable the direct search of concrete
counter-examples. We remedy this by incorporating the estimated local
dynamics by annotating the graph edges with them. The resulting
transition system is rich enough that an off-the-shelf bounded model
checker can find concrete violations of a safety property, thus doing
away with the expensive refinement process. We now discuss the
background required to present our ideas.


\subsection{Relational Abstraction}

For continuous dynamical systems, it is hard to directly reason about
reachability. Discretization is often employed to transform the
systems into a discrete transition system. One idea is to abstract
continuous relations by \textit{reachability invariants}. The
resulting relations can be either time independent or time dependent.
Depending on the property at hand, we can select the appropriate one.
The resulting abstractions (usually discrete transitions systems) can
be analyzed using model checkers. We briefly discuss the two types of
relational abstractions.

\mypara{Time-less Relational Abstraction}

Relational abstractions for hybrid systems were proposed in
\cite{Sankaranarayanan+Tiwari/2011/Relational}. Proposed for hybrid
automaton models, the idea is to summarize the continuous dynamics of
each mode using a relation over reachable states. The resulting
relation is time independent and hence valid for all time as long as
the mode invariant is satisfied. The relations take the general form
of $R(\x,\x') \bowtie 0$, where $\bowtie$ represents one of the
relational operators $=, \ge, \le, <, >$. For \eg, an abstraction
which captures the monotonicity with respect to time for the
differential equation $\dot{x} = 2$ is $x' > x$.  The abstraction
capturing the relation between the set of ODEs: $\dot{x} = 2$,
$\dot{y} = 5$, is $5(x' - x) = 2(y' - y)$.

Such relations summarize the given system as a transition system.
These can be verified using $k$-induction or falsified using bounded
model checkers. The relationalization procedure involves finding
suitable abstractions, such as affine abstractions, eigen abstractions
and box abstractions~\cite{Sankaranarayanan+Tiwari/2011/Relational}.
For this, differnt techniques like template based invariant
generation~\cite{Gulwani+Tiwari/2008/Constraint,
Colon+Sankaranarayanan+Sipma/03/Linear} can be used.

\mypara{Timed Relational Abstraction}

As the above discussed relations are timeless, we cannot check their
timing properties.  Moreover, they cannot be easily specialized for
time-triggered systems. Timed relational abstractions provide a
similar solution by computing relations, but, time dependent ones.

For the case of dynamics defined by affine set of ODEs, timed
relations are provided by their solutions $\x(t) = e^{tA}\x(0)$.
This gives us the timed relation $\x' = e^{tA}\x$. Clearly, the
relation is non-linear with respect to time. However, for the case of
SDCS with a fixed time period, we obtain linear relations in $\x$.
We demonstrated the usefulness of timed relational abstractions on
linear systems in~\cite{zutshi2012timed}, and now incorporate the idea
to discretize black box dynamical systems.

%TODO
% \subsection{Time-Aware}
% Captures the behavior between two time instants which timed relational
% abstraction ignores.

\subsection{Simple Linear Regression}

\mypara{Regression}

In statistics, regression is the problem of finding a
\textit{predictor}, which can suitably predict the relationship
between the given set of observed input $\x$ and output $\y$ vectors.
In other words, assuming that $\y$ depends on $\x$, regression
strategies find either a parameterized or a non-parameterized
prediction function to explain the dependence. We now discuss simple
linear regression, which is parametric in nature and searches for an
affine predictor. It is also called \textit{ordinary least squares}.

\mypara{Ordinary Least Squares (OLS)}

Let the data set be comprised of $N$ input and output pairs $(\x,
y)$, where $\x\in\reals^n$ and $y\in\reals$. If $N>n$, which is
the case in the current context of \textit{finding the best fit}, the
problem is overdetermined; there are more equations than
unknowns. Hence, a single affine function cannot be found which
satisfies the equation $\forall i\in\{1\ldots N\}. y_i = A\x_i + \vb$. Instead, we
need to find the `best' choice for $A$ and $\vb$. This is formally
defined using a loss function. For the case of simple linear
regression or OLS, the loss function is the sum of squares of the
errors in prediction.  The task is then to determine the matrix of
coefficients $A$ and an offset vector $\vb$, such that the least
square error of the affine predictor is minimized for the given data
set.
\[
    \min_{A, b}\displaystyle\sum_{i=1}^{N}{\left(\y_i - (A\x_i + \vb)\right)^2}
\]
The solution of OLS can be analytically computed as
\[ A = (X^TX)^{-1}X^T\y\]
where $X$ is the matrix representing the horizontal stacking of all $\x$.
The details can be found in several texts on learning and statistics
\cite{friedman2001elements}.

\subsection{Piecewise Affine Model (PWA)}

A PWA model is a collection of guarded affine maps which compute the
next state of the underlying dynamical system. The affine maps
discretize the underlying continuous-time dynamics.

Given a state vector $\x\in\reals^n$, a guarded affine map
$\gm:(\guard, \amap)$ defines the discretized consecution rule as a
pair of an affine guard predicate $\guard:C\x - \vd \le 0$ and an affine map
$\amap:A\x+\vb$, where $A \in \reals^{n \times n}$, $C \in
\reals^{m \times n}$ are matrices and $\vb \in \reals^n$, $\vd
\in \reals^m$ are vectors. A guarded affine map is satisfied if its
guard is satisfied.

We now formalize the PWA affine model for a dynamical system.
\begin{definition}[PWA Model]

Given a dynamical system over a continuous state-space
$\ContStates\in\reals^n$, a PWA model is a map $\ContStates \mapsto
\setof{\gm_1, \gm_2, \ldots, \gm_n}$ from the state-space of the
dynamical system to a finite set of guarded affine maps
$\setof{\gm_1, \gm_2, \ldots, \gm_n}$. It defines the consecution
rule by the affine map of the satisfied $\gm$.

\begin{equation}
    \pwa = \left\{
        \begin{array}{ll}
            \gm_1: \amap_1(\x) &\text{if}\; \guard_1(\x) \\
            \gm_2: \amap_2(\x) &\text{if}\; \guard_2(\x) \\
            \ldots & \ldots\\
            \gm_n: \amap_n(\x) &\text{if}\; \guard_n(\x) \\
        \end{array}
    \right.
\end{equation}

\end{definition}

where $\amap_i = A_i\x + \vb_i$, and $\guard_i(\x) \equiv C_i\x-\vd_i\le0$.
Abusing the notation, we denote the next state computed by the PWA
model as $\x' = \pwa(\x)$. A PWA model is \textit{deterministic}, iff
for every state $\x\in\ContStates$, a unique guarded affine map is
satisfied. A PWA model is \textit{complete}, iff for every state
$\x\in\ContStates$, there exists at least one satisfied guarded affine
map $\gm$.
% $\forall x \in \ContStates. \forall \gm_i. $

\include{segtrajs}
\include{abstractions}
\include{symex}

\section{Implementation and Evaluation}

The implementation was prototyped as S3CAM-R, an extension to our
previously mentioned tool S3CAM (\chapref{case}). OLS regression
routines were used from Scikit-learn~\cite{pedregosa2011scikit},
Python module for machine learning. SAL~\cite{SAL-SRI}
with Yices2~\cite{dutertre2014yices} was the model checker.

We tabulate our preliminary evaluation in \tabref{res-rel}. We use the
Van der Pol oscillator and the Brusselator, as described in the previous
chapter. As before, we ran S3CAM-R $10$ times with different seeds and
averaged the results. We tabulate both the total time taken and the
time taken by SAL to compute the counter-example and compare against
S3CAM.

\begin{table*}[!htbp]
\centering
\caption{Avg. timings for benchmarks. The \textbf{BMC} column lists time
    taken by the BMC engine. The total time is noted under
\textbf{S3CAM-R} and \textbf{S3CAM}.}
\label{tab:res-rel}
\begin{tabular}{@{}llll@{}}
\toprule
Benchmark & BMC & S3CAM-R & S3CAM\\
\midrule
Van der Pol ($\scr{P}3$)    & $0.0s$  & $10.0s$ & $15.0s$\\
Brusselator               & $0.2s$ & $4.0s$  & $2.5s$\\
%Lorenz                    & $0.$ & $s$  & $.s$\\
\bottomrule
\end{tabular}
\end{table*}

% 1m8s, 2m56.184s, 1m31.059s[f], 1m32.665s,  1m45.312s, 1m22.397s, 2m1.451s[f], 1m12.118s, 1m26.596s
% 0.12, 0.06,         0.33       0.13,       0.22,     FAIL,       ,X           0.08        0.15

The results are favorable and show promise. However, we need to
explore more benchmarks to conclude conclusively.

\include{relational}


\section{Summary}% and Future Work}

We have presented another methodology to find falsifications in black
box dynamical systems. Combining the ideas from abstraction based
search (\chapref{abs}), with our previous relational
abstractions~\cite{zutshi2012timed}, we created enriched abstractions.
These can be checked for safety violations using existing bounded
model checkers.  We used learning techniques to estimate the local
dynamics underlying trajectory segments, and approximated a transition
system from a black box system. Finally, we showed our approach on a
few examples.

As a future extension, we are working on a specialized BMC for the
problem at hand. We intend to explore efficient ways for encoding the
PWA relational system.

%  There are several directions we can take from here.

%  \paragraph{Impr}The biggest impediment to our approach are SMT solvers which use the
%  theory of reals with exact precision. This is important for
%  verification approaches, but not can be relaxed for falsification. An
%  SMT solver which uses approximate reasoning but returns robust
%  counter-examples will be very useful, though it doesn't exist as yet.
%  
%  Specializing BMC towards discretization
%  
%  \subsection{Challenges and Extensions}
%  Let us outline the challenges we face in this approach. Although
%  SAT/SMT solvers are very efficient owing to extensive engineering
%  effort, reachability of transition systems resulting from dynamical
%  systems remains a difficult problem.
%  
%  %Why exactly?
%  As the time horizon of the safety property increases, the possible
%  combinations of discrete transitions increases exponentially. Hence,
%  to find a counterexample which is a sequence of discrete transitions
%  over a `long' time horizon is not tractable for most but the simplest
%  of dynamical systems. % Solve using proper discretizations

%\section{Learning Discrete Abstraction for Black Box Systems}
%\section{Comparison with traditional Trajectory Linearization}
%\subsection{Piece-Wise Affine Model}
%\subsection{Other models}


%%%%%%%%%   then the Bibliography, if any   %%%%%%%%%
\bibliographystyle{plain}	% or "siam", or "alpha", etc.
%\nocite{*}		% list all refs in database, cited or not
\bibliography{refs}		% Bib database in "refs.bib"

%%%%%%%%%   then the Appendices, if any   %%%%%%%%%
% \appendix
% \include{appendixA.tex}
% \include{appendixB.tex}

\end{document}
