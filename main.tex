%\documentclass{sig-alternate}
\documentclass{sig-alternate-05-2015}
\usepackage{tikz}
\usepackage{framed}
\usepackage{stmaryrd}
\usepackage{amsmath}
\usepackage{ragged2e}
\usepackage{varwidth}

\usetikzlibrary{arrows,backgrounds,positioning,fit,automata,shapes,snakes,patterns}
\usetikzlibrary{arrows,backgrounds,decorations,decorations.pathmorphing,positioning,fit,automata,shapes,snakes,patterns}
\usetikzlibrary{shapes.geometric, arrows, positioning, calc, matrix}
\tikzset{block/.style = {draw, fill=blue!20, rectangle,
                         minimum height=3em, minimum width=6em},
        sum/.style = {draw, fill=blue!20, circle, node distance=1cm},
        input/.style = {coordinate},
        output/.style = {coordinate},
        pinstyle/.style = {pin edge={to-,thin,black}}
}

\usepackage{balance}
\usepackage{cite}

\makeatletter
\newif\if@restonecol
\makeatother
\let\algorithm\relax
\let\endalgorithm\relax

\renewcommand{\baselinestretch}{0.97}

\usepackage{times}
%\usepackage{graphicx,enumerate}
\usepackage{graphicx}
\usepackage{enumitem}
%\usepackage{enumerate}
\usepackage{subfig}
%\usepackage{caption}
%\usepackage{subcaption}
\usepackage{amsmath,amssymb,amsfonts}
\usepackage{fancyhdr}
\usepackage{balance}
\usepackage{cite}
\usepackage{color}
\usepackage{wrapfig}
\usepackage[ruled,vlined,linesnumbered]{algorithm2e}
%\usepackage{booktabs}
\usepackage{booktabs,siunitx}
\usepackage{xcolor,colortbl}
\usepackage{times}
\usepackage{microtype}
\usepackage{url}
\usepackage{listings}
\usepackage{multirow}
%\usepackage{tablefootnote}

% \usepackage{hyperref}
% \hypersetup{
%     colorlinks=false,
%     pdfborder={0 0 0},
% }

%%\usepackage{abstract}

\makeatletter
\def\url@mystyle{%
  %\@ifundefined{selectfont}{\def\UrlFont{\sf}}{\def\UrlFont{\small\ttfamily}}}
  \@ifundefined{selectfont}{\def\UrlFont{\mathtt}}{\def\UrlFont{}}}
  %\@ifundefined{selectfont}{\def\UrlFont{\sf}}{\def\UrlFont{\small}}}
\makeatother
%\urlstyle{mystyle}
\urlstyle{rm}


\definecolor{mygreen}{rgb}{0,0.6,0}
\definecolor{mygray}{rgb}{0.5,0.5,0.5}
\definecolor{mymauve}{rgb}{0.58,0,0.82}

\newtheorem{example}{Example}[section]
\newtheorem{definition}{Definition}[section]
\newtheorem{assumption}{Assumption}[section]
\newtheorem{lemma}{Lemma}[section]
\newtheorem{theorem}{Theorem}[section]

%TODO: Fix line 738 of sig-alternate-05-2015.cls for the copyright
%notice to appear
\DeclareCaptionType{copyrightbox}

% Generic
%\DeclareMathAlphabet{\mathpzc}{OT1}{pzc}{m}{it}

% English
\newcommand{\ie}{{i.e.}\xspace}
\newcommand{\Ie}{{I.e.}\xspace}
\newcommand{\eg}{{e.g.}\xspace}
\newcommand{\etc}{{etc.}\xspace}
\newcommand{\viz}{{viz.\xspace}}
\newcommand{\etal}{{et al.}\xspace}

\renewcommand\vec[1]{\mathbf{#1}}
% Generic refs
\newcommand{\lemref}[1]{Lemma~\ref{lem:#1}}
\newcommand{\secref}[1]{Sec.~\ref{sec:#1}}
\newcommand{\figref}[1]{Fig.~\ref{fig:#1}}
\newcommand{\exref}[1]{Example~\ref{ex:#1}}
\newcommand{\thmref}[1]{Theorem~\ref{thm:#1}}
\newcommand{\tabref}[1]{Table~\ref{tab:#1}}
\newcommand{\algoref}[1]{Algorithm~\ref{algo:#1}}
\newcommand{\lstref}[1]{Listing~\ref{lst:#1}}
\newcommand{\chapref}[1]{Chapter~\ref{chap:#1}}
\newcommand{\defref}[1]{Definition~\ref{def:#1}}

% Comments, Reviewing, Formatting
\newcommand{\ignore}[1]{}
\newcommand\todo[1]{[[\textcolor{red}{\textsf{#1}}]]}
% General Math
\newcommand{\setof}[1]{\ensuremath{\{#1\}}}
\newcommand\tupleof[1]{\ensuremath\left\langle #1 \right \rangle}
\newcommand{\sublist}[2]{\ensuremath{#1_{1},\ldots,#1_{#2}}}
\newcommand{\suplist}[2]{\ensuremath{#1^{1},\ldots,#1^{#2}}}
\newcommand \card[1] {\left| #1 \right|}
\newcommand \floor[1] {\left\lfloor #1 \right\rfloor}
\newcommand \ceil[1] {\left\lceil #1 \right\rceil}

% Complexity
\newcommand{\npcomplete}{\textsc{Np}-\textsc{complete}}
\newcommand{\nphard}{\textsc{Np}-\textsc{Hard}}
\newcommand{\nlogspace}{\textsc{NLogspace}\xspace}
\newcommand{\pspace}{\textsc{PSpace}\xspace}
\newcommand{\expspace}{\textsc{ExpSpace}\xspace}

\newcommand{\mcl}[1]{\multicolumn{1}{l}{#1}}
\newcommand{\mcc}[1]{\multicolumn{1}{c}{#1}}

\newcommand{\mypara}[1]{\vspace{0.6em} \noindent{\bf #1}}
\newcommand{\myipara}[1]{\vspace{0.6em} {\em #1}\xspace}
%\newcommand{\myipara}[1]{\vspace{0.6em} \noindent{\em #1}\xspace}
\newcommand{\myiparaCompact}[1]{ {\em #1}\xspace}

% Sets
\newcommand{\Reals}{\ensuremath{\mathbb{R}}}
\newcommand{\reals}{\Reals}
\newcommand{\Nats}{\ensuremath{\mathbb{N}}}
\newcommand{\Integers}{\ensuremath{\mathbb{Z}}}
\newcommand{\Rationals}{\ensuremath{\mathbb{Q}}}

%%\newcommand{\mathsf}[1]{\mbox{\textsc{#1}}}
\newcommand{\mathsc}[1]{\mbox{\sc #1}}
\newcommand\scr[1]{\ensuremath\mathcal{#1}}
\newcommand\Gradient{\ensuremath\nabla}
\newcommand\pdiff[2]{\partial_{#1}{#2}}
\newcommand\jth[2]{ #1^{(#2)} }

% Logic
% \newcommand{\land}{\ensuremath\wedge}
% \newcommand{\lor}{\ensuremath\vee}

% Software
\newcommand\flowstar{FLOW*}
\newcommand\MATLAB{Matlab\textsuperscript{\textregistered}}
\newcommand\SIMULINK{Simulink\textsuperscript{\textregistered}}
\newcommand\STATEFLOW{Stateflow\textsuperscript{\textregistered}}
\newcommand\EMCODER{Embedded Coder\textsuperscript{\textregistered}}

\newcommand\vs{\mathbf{s}}
\newcommand\vw{\mathbf{w}}
\newcommand\vx{\mathbf{x}}
\newcommand\vy{\mathbf{y}}
\newcommand\vz{\mathbf{z}}
\newcommand\vu{\mathbf{u}}
\newcommand\vj{\mathbf{j}}

% long squiggily arow
\newcounter{sarrow}
\newcommand\xrsquigarrow[1]{%
\stepcounter{sarrow}%
\begin{tikzpicture}[decoration=snake]
\node (\thesarrow) {\strut#1};
\draw[->,decorate] (\thesarrow.south west) -- (\thesarrow.south east);
\end{tikzpicture}
}

% misc
\newcommand\ii{i+1}
\newcommand\denotation[1]{ \left\llbracket #1 \right\rrbracket}



% The \munepsfig command is used to insert a new EPS figure
% into our document.  Usage is:
%
%       \munepsfig[args]{filename}{caption}
%
% where:
%       - the optional 'args' argument is passed to the
%         embedded \includegraphics command, this can be used
%         to scale the figure or rotate it.
%       - 'filename' is the name of the EPS file in the 'figures'
%         directory that is to be inserted (note that 'filename'
%         should not include the '.eps' extension).
%       - 'filename' also serves as the label for the figure.
%         with the text 'fig:' prepended.
%
% Sample Usage:
%       \munepsfig[scale=0.5,angle=90]{barchart}{Population over time}

% inserts the EPS file 'figures/barchart.eps' reduced in size by 50%
% rotated 90 degrees and with the caption "Popuation over Time."
% We can refer to that figure as Figure~\ref{fig:barchart} in the text.
%
\newcommand{\inclfig}[3][scale=1.0]{%
        \begin{figure}[!htbp]
                \centering
                \vspace{2mm}
%               \includegraphics[#1]{figures/#2.eps}
                \includegraphics[#1]{figs/#2}
                \caption{#3}
                \label{fig:#2}
        \end{figure}
}

% Units
%\newcommand{\degree}{^{\circ}}
\newcommand{\degreeC}{^{\circ}{\rm C}}
%\newcommand{\degreeF}{^{\circ}{\rm F}}

%\newcommand{\CC}{C\nolinebreak\hspace{-.05em}\raisebox{.4ex}{\tiny\bf +}\nolinebreak\hspace{-.10em}\raisebox{.4ex}{\tiny\bf +}}
\def\Cpp{{C\nolinebreak[4]\hspace{-.05em}\raisebox{.4ex}{\tiny\bf ++}}}
\def\CC{{C\nolinebreak[4]\hspace{-.05em}\raisebox{.4ex}{}}\;}



% RRT
\newcommand{\RRT}{\mathcal{RRT}}
\newcommand{\RRTgoal}{\x_{goal}}
\newcommand{\RRTinit}{\x_{init}}
\newcommand{\RRTv}[1]{v_{#1}}
\newcommand{\RRTe}{e}
\newcommand{\RRTV}{V}
\newcommand{\RRTE}{E}
\newcommand{\RRTsample}{x_{sample}}
\newcommand{\RRTnear}{x_{near}}
\newcommand{\RRTtoEx}{x_e}
\newcommand{\RRTxNew}{x_{new}}
\newcommand{\RRTuNew}{u_{new}}
\newcommand{\RRTdedge}[3]{(#1 \xrightarrow{#3} #2)}

% Graph
\newcommand{\Graph}{\mathbf{G}}
\newcommand{\graph}{\mathbf{G}}
\newcommand{\vertex}[1]{v_{#1}}
%\newcommand{\vertex}[1]{#1}
\newcommand{\dedge}[2]{e_{(\vertex{#1}\rightarrow\vertex{#2})}}
\newcommand{\edge}{e}
\newcommand{\vertexSet}{V}
\newcommand{\edgeSet}{E}
%\newcommand{\vertexSetD}{V^\delta}
%\newcommand{\edgeSetD}{E^\delta}
%\newcommand{\graphVE}[1]{\mathbf{G_{#1}(\vertexSet,\edgeSet)}}
%\newcommand{\graphD}[1]{\mathbf{G^\delta_{#1}(\vertexSet^\delta,\edgeSet^\delta)}}
\newcommand{\kPaths}{k\_paths}
\newcommand{\trajToNode}{\Gamma}
\newcommand{\Path}{\mathbf{P}}
\newcommand{\weight}{\mathbf{W}}

% Hybrid Automata
\newcommand{\HA}{\ensuremath{\mathcal{A}}}

\newcommand{\System}{\ensuremath{S}}
\newcommand{\Inputs}{\ensuremath{\mathcal{U}}}
\newcommand{\Flow}{\ensuremath{\mathcal{F}}}
\newcommand{\Modes}{\ensuremath{\mathcal{Q}}}
\newcommand{\ContStates}{\ensuremath{X}}
\newcommand{\Inv}{\ensuremath{\mathcal{I}}}
%\newcommand{\Transitions}{\ensuremath{\Delta}}
\newcommand{\Transitions}{\scr{T}}
\newcommand{\HybridStates}{\ensuremath{\mathcal{X}}}
\newcommand{\HybridStateSet}{\ensuremath{X}}
\newcommand{\ResetMap}{\mathcal{R}}
\newcommand{\Guards}{\mathcal{G}}
\newcommand{\Init}{\mathcal{X}_{0}}
\newcommand{\Err}{\ContStates_{f}}
\newcommand{\initmode}{m_{init}}
\newcommand{\cInit}{\ensuremath{X}_0}
\newcommand{\Unsafe}{\mathcal{X}_f}
%\newcommand{\reachSet}[2]{R^{#1}_{\HA}({#2})}
%\newcommand{\Hflow}{\mathcal{H}_{\HA}}
\newcommand{\Hflow}{\mathcal{H}}
\newcommand{\reachSet}{R}

% Trajectories
\newcommand{\discTraj}{q_h}
\newcommand{\hybridTraj}{\tau_h}
\newcommand{\traj}{\pi}
\newcommand{\trajSeg}[2]{\pi^{\mode_{#1}}_{\tau_{#2}}}
\newcommand{\SegTraj}{\mathbf{S}_\pi}
\newcommand{\dwell}{\tau}
\newcommand{\tran}{\delta}
\newcommand{\tbegin}{b}
\newcommand{\tend}{e}
%\newcommand{\graph}{G}
\newcommand{\trajStore}{TS}
\newcommand{\trajSet}{TS}
\newcommand{\candTraj}{CT}
\newcommand{\candTrajSet}{CTS}
\newcommand\Cost{\mathsc{Cost}}


% State variables
\newcommand{\x}{\mathbf{x}}
\newcommand{\y}{\mathbf{y}}
\newcommand{\z}{\mathbf{z}}
%\newcommand{\u}{\mathbf{x}}
\newcommand{\w}{\mathbf{w}}
%\newcommand{\vec}[1]{\mathbf{#1}}
\newcommand{\dvx}{\mathbf{\dot{x}}}
\newcommand{\inp}{\mathbf{u}}
\newcommand{\mode}{\ensuremath{q}}
\newcommand{\dx}{\ensuremath{\dot{x}}}

% Metrics
\newcommand{\lmetric}[2]{{\Vert #2 \Vert}_#1}

% ODE
\newcommand{\odesolution}{\Phi}
\newcommand{\matA}{\mathbf{A}}
\newcommand{\matB}{\mathbf{B}}
\newcommand{\mata}{\mathbf{a}}


\newcommand \maximize{\mathbf{max.}\ }
\newcommand \minimize{\mathbf{min.}\ }

\newcommand \numsolve{\mathtt{NumericSolver}}
\newcommand\Flowmap{\mathsc{Flow}}
%%\newcommand\myipara[1]{\par\noindent\textit{#1:}}

% Arrows
%\newcommand{\contArrow}[2]{\underset{#2}{\overset{#1}{\leadsto}}}
\newcommand{\contArrow}[1]{\leadsto_{#1}}
\newcommand{\jumpArrow}[1]{\xrightarrow{#1}}


\newcommand\dist{\mathsf{d}}
\newcommand\src{\mathsf{src}}
\newcommand\dest{\mathsf{dest}}
\newcommand\timeElapse{\mathcal{T}}
\newcommand\simulate{\mathsc{sim}}
%%\newcommand\sim{\mathsc{sim}}
\newcommand\cost{\mathsc{cost}}
\newcommand\reach[1]{\xrightarrow{#1}}
\newcommand\areach[1]{\overset{#1}{\rightsquigarrow}}
%\newcommand\areach[1]{\xrsquigarrow{#1}}

%\newcommand\areach[2]{\overset{#1}{\underset{#2}\rightsquigarrow}}

\newcommand\crel[1]{\xrightarrow{#1}}
\newcommand\drel[1]{\overset{#1}{\rightsquigarrow}}
\newcommand\rel[1]{\overset{#1}{\rightsquigarrow}}


\newcommand\intr{\mathsf{interior}}
\newcommand\assign{:= }
\newcommand\worklist{\mathsf{workList}}
\newcommand\exploredCells{\mathsf{V}}
\newcommand\exploredEdges{\mathsf{E}}
\newcommand\unsafeCells{\mathsf{V}_u}
\newcommand\initialCells{\mathsf{V}_0}
\newcommand\prb{\mathbb{P}}
\newcommand{\abstracteps}{\epsilon}
\newcommand{\refinedeps}{\delta}
\newcommand{\absgraph}{\scr{H}_{\epsilon}(\Delta)}
\newcommand{\refgraph}{\scr{H}_{\delta}(\Delta)}
\newcommand{\abscells}{{\scr{C}}}
\newcommand{\refcells}{{\scr{D}}}






%%%%%%%% from Sriram sty, rearrange!

\def\mathsc#1{\mbox{\sc #1}}

\newcommand\trp[1] {#1^{\scriptscriptstyle T}}
\newcommand\scrS{\mathcal{S}}
\newcommand\scrT{\mathcal{T}}
\newcommand\scrF{\mathcal{F}}
\newcommand\scrG{\mathcal{G}}
\newcommand\scrH{\mathcal{H}}
\newcommand \ints {\ensuremath \mathbb{Z}}
%%\newcommand \extreals {\ensuremath \mathcal{R}^{+}}
\newcommand \lin[1]{\mathit{Lin}(#1)}
\newcommand \conic[1]{\mathit{Cone}(#1)}
\newcommand \conv[1]{\mathit{Convex}(#1)}
\newcommand \false {\mathit{false}}
\newcommand \true  {\mathit{true}}
%\newcommand\pre{\preceq}
\newcommand \pres{\mathbf{pres}}
\newcommand \F {\mathfrak{F}}

\newcommand \abstractF {\F_A}
\newcommand \abstractDomain {\Sigma_A}
\newcommand \templateDomain {\Sigma_T}
\newcommand \dropped {\mathsc{x}}
\newcommand \ctop {\vec{c}_\top}
\newcommand \cbot {\vec{c}_\bot}
\newcommand \concT {\gamma_T}
\newcommand \abstractorder {\leq_A}
\newcommand \abstractLeq{\sqsubseteq}
\newcommand \abstractor {\sqcup}
\newcommand \abstractand {\sqcap}
\newcommand \bigabstractor {\bigsqcup}
\newcommand \setdef[2] { \left\{ #1 \mid #2 \right\}}

\newcommand \CH {\mathcal{CH}}


%%\newcommand \mathsc[1]{\mbox{\textsc{#1}}}
%%transpose



\newcommand \T {\mathcal{T}}
\newcommand \post {\ensuremath\mathit{post}}
\newcommand \dpost {\widehat{\mathit{post}}}
\newcommand \homg[1] {\mathsc{hom}(#1)}
\newcommand \cons{\mathsc{cons}}
\newcommand \widen {\nabla}
\newcommand \dual[1]{\widehat{#1}}
\newcommand \narrow {\/ \bigtriangleup \/ }
\newcommand \refine {\partial}
\newcommand \deta{\pi}

\newcommand \latcap {\sqcap}
\newcommand \latcup {\sqcup}
%\newcommand \guard {\xi}
\newcommand \templ{\gamma}
\newcommand \sizeof[1] { |#1| }
\newcommand \uset[2]{\underset{#1}{\underbrace{#2}}}
\newcommand \ith[2]{ {#1}^{(#2)}}
\newcommand \C {\ensuremath \mathcal{C}}
\newcommand \complex {\ensuremath \mathcal{C}}
\newcommand \grb {Gr\"obner\xspace}
\newcommand \hi[1]{}
\newcommand \hihenny[1]{}
\newcommand \newhenny[1]{#1}
\newcommand\markmargin[2]
{[\marginpar[\hfill \mbox{#1}$\rightarrow$]{$\leftarrow$\mbox{#1}} 
{\sf #2]}}

\newcommand\highl[1]{\psframebox[fillstyle=solid,fillcolor=lightgray,linewidth=0pt]{\textbf{#1}}}


\newcommand \locs{\mathbf{L}}
\newcommand\en{\mathbf{en}}




\newcommand{\chapterheading}[1]
{\vfill  
\hfill \fbox{\begin{minipage}{5.5in}
\textsl{#1} \end{minipage} } \newpage}

\newcommand \Z{ \mathbf{Z}}

\newcommand \mand {\ \mathit{and}\ }
\newcommand \matb{\vec{b}}
\newcommand \matl{\vec{\lambda}}
\newcommand \vg {\vec{g}}
\newcommand \va{\vec{a}}
\newcommand\ve{\vec{e}}
\newcommand \vc {\vec{c}}
\newcommand \vf {\vec{f}}
\newcommand \vh {\vec{h}}
\newcommand \vv{\vec{v}}
\newcommand \vzero {\vec{0}}
\newcommand \vq {\vec{q}}
\newcommand \rank {\mathit{rn}}
\newcommand\vbeta{\vec{\rho}}
\newcommand \Petri {\ensuremath \mathcal{P}}
\newcommand \maxrank {\mathit{maxrank}}
\newcommand \lists{\mathit{list}}
\newcommand \nullsp{\mathit{null}} 
\newcommand \vl[1] {\vec{\lambda_{#1}}}
\newcommand\vlam{\vec{\lambda}}
\newcommand \st{\mathbf{s.t.}\ }


\newcommand \D {\mathit{D}}
\newcommand \I {\mathit{I}}
\newcommand \Loc{\mathit{Loc}}
\newcommand \lie{\mathcal{L}}
\newcommand \grad{\nabla}
\newcommand \cn {\mathsf{Cn}}

%\newcommand \diff[2] {\ensuremath \frac{d #1}{d #2}}

\renewcommand\paragraph[1] {\smallskip\par\noindent\textbf{#1}\ \ }
\newcommand \cplus {\uplus}

\newcommand\relop{\mathop{\bowtie}}
\newcommand \vt{\vec{t}}
\newcommand \vd{\vec{d}}
\newcommand\e{\mathsf{e}}
\newcommand\yl{\mathsf{l}}
\newcommand\yu{\mathsf{u}}
\newcommand\yb{\mathsf{b}}


\newcommand\init{\mathsf{init}}

\newcommand\ol[1]{\overline{#1}}
\newcommand\state[1]{\vec{{\texttt{#1}}}}

\newcommand\scrP{\mathcal{P}}
\newcommand\scrX{\mathcal{X}}
\newcommand\scrC{\mathcal{C}}
\newcommand\interVal[1]{[\underline{#1}, \overline{#1}]}
\newcommand\HIDE[1]{}


\newcommand{\HErr}{\mathcal{X}_{f}}
\newcommand\Eq{\mathsf{Eq}}
\newcommand\Ineq{\mathsf{Ineq}}
%\newcommand \setof[1] { \left\{ #1 \right \}}
%\newcommand \vx {\vec{x}}
%\newcommand \vy {\vec{y}}
%\newcommand \vz {\vec{z}}
%\newcommand \vu {\vec{u}}
\newcommand \vb{\vec{b}}
%\newcommand \pdiff[2] { \ensuremath \frac{\partial #1}{\partial #2}}
%\newcommand \reach[1]{\mathsf{Reach}(#1)}


\newcommand{\vertexSetD}{V^\delta}
\newcommand{\edgeSetD}{E^\delta}
\newcommand{\graphVE}[1]{\mathbf{G_{#1}(\vertexSet,\edgeSet)}}
\newcommand{\graphD}[1]{\mathbf{G^\delta_{#1}(\vertexSet^\delta,\edgeSet^\delta)}}



% plant and controller

\newcommand\pgm{\rho}
%\newcommand\outputs{Y}
%\newcommand\locs{L}
%\newcommand\sloc{l_i}
%\newcommand\eloc{l_o}
%\newcommand\op{op}
%\newcommand\expr{E}


\newcommand\symMem{\mu}
\newcommand\CFG{\Pi}
\newcommand\pLocs{L}
\newcommand\pEdges{E}
\newcommand\pLabels{\Phi}
\newcommand\pEdge[1]{edge(#1)}
\newcommand\pVars{\mathcal{V}}
\newcommand\pVar{v}
\newcommand\pTransRel{\rho}
\newcommand\pLoc{l}
\newcommand\pLocI{l_0}
\newcommand\pLocF{l_f}
\newcommand\pVarI{V_0}
\newcommand\pPaths{\mathcal{P}}
\newcommand\pPath{p}
\newcommand\pPathCons{\kappa}
\newcommand\pCons{\xi}
\newcommand\pSymMem{\sigma}
\newcommand{\domain}{\mathcal{D}}
\newcommand{\pre}{\mathit{pre}}
\newcommand{\update}{\mathit{update}}
%TODO: redef?? fixit!
%\newcommand{\C}{\mathtt{C}}

\newcommand{\sampletime}{{\tau_{s}}}

\newcommand\pgmF{\rho}
\newcommand\pgmA{\hat{\rho}}
\newcommand\plantStates{x}
\newcommand\abstractPlantStates{X}
\newcommand\controllerOutputs{u}
\newcommand\controllerStates{s}
\newcommand\PathCons{\kappa}
\newcommand\constraints{\psi}
%TODO: redinfe path...conflicting with prev def
%\newcommand\Path{\mathcal{P}}
\newcommand\absState{\mathcal{A}}

\newcommand\ts{\tau_s}

\newcommand\Cix{C_i^x}
\newcommand\Ciy{C_i^y}
\newcommand\Ciix{C_{i+1}^x}
\newcommand\Ciiy{C_{i+1}^y}
% \newcommand\si{s_i}    %conflicts with package siuntix
\newcommand\sii{s_{i+1}}
\newcommand\ui{u_i}
\newcommand\uii{u_{i+1}}
\newcommand\sip[1]{s_{i}^{p_{#1}}}
\newcommand\uip[1]{u_{i}^{p_{#1}}}
\newcommand\siip[1]{s_{i+1}^{p_{#1}}}
\newcommand\uiip[1]{u_{i+1}^{p_{#1}}}
\newcommand\kpi{\kappa^{p_i}}


\renewcommand{\Si}{S_i}
%\newcommand\Si{S_i}
\newcommand\Sii{S_{i+1}}
\newcommand\Ui{U_i}
\newcommand\Uii{U_{i+1}}
\newcommand\Sip[1]{S_{i}^{p_{#1}}}
\newcommand\Uip[1]{U_{i}^{p_{#1}}}
\newcommand\Siip[1]{S_{i+1}^{p_{#1}}}
\newcommand\Uiip[1]{U_{i+1}^{p_{#1}}}

\newcommand \ctrl{\mathsc{ctrl}}
\newcommand\sample{\mathsc{Sample}}

\newcommand\TrajOpt{TrajOpt\;}

\newcommand{\cellequivalence}{\equiv_{\scr{C}}}
\newcommand{\vk}{\mathbf{k}}



\newcommand{\Traj}{\tau}
\newcommand{\cstates}{\scr{X}}
\newcommand{\csys}{S}
\newcommand{\cell}{C}
\newcommand{\quant}{Q}
%\newcommand\quant{\mathsc{Quant}}

\newcommand\fmincon{\texttt{fmincon}\xspace}


\newcommand{\pwa}{\rho}
\newcommand{\map}{f}
\newcommand{\amap}{f}
\newcommand{\gm}{\scr{T}}
\newcommand{\guard}{g}
\newcommand{\ds}{\scr{D}}


%\clubpenalty=10000
%\widowpenalty = 10000

\title{Symbolic-Numeric Reachability Analysis of Closed-Loop Control Software}

\numberofauthors{3}
\author{
\alignauthor
Aditya Zutshi\\
\affaddr{\small{University of Colorado, Boulder}}\\
\email{\small{aditya.zutshi@colorado.edu}}\\
    \and
\alignauthor
Sergio Mover\\
\affaddr{\small{University of Colorado, Boulder}}\\
\email{\small{sergio.mover@colorado.edu}}
  \and
\alignauthor
Sriram Sankaranarayanan\\
\affaddr{\small{University of Colorado, Boulder}}\\
\email{\small{srirams@colorado.edu}}
}

\date{\today}

\begin{document}

%\CopyrightYear{2016}
%\setcopyright{acmcopyright}
%\conferenceinfo{HSCC'16,}{April 12-14, 2016, Vienna, Austria}
%\isbn{978-1-4503-3955-1/16/04}\acmPrice{\$15.00}
%\doi{http://dx.doi.org/10.1145/2883817.2883819}

\maketitle

\begin{abstract}
    In this work we address the problem of findfing behaviors violating a
given time bounded LTL property in hybrid dynamical systems. We
approach the problem in two distinct steps. First, we approximate the
behavior of the underlying system using a piecewise affine discrete
time model. The model is incomplete in nature and is computed with
respect to the given property.  We then encode the falsification
search as a bounded model checking query and use an SMT solver to find
a counter example. If found, we check the violation to ensure
reproducibility in the given hybrid dynamical system.

\end{abstract}

%
% The code below should be generated by the tool at
% http://dl.acm.org/ccs.cfm
% Please copy and paste the code instead of the example below.
%
\begin{CCSXML}
<ccs2012>
<concept>
<concept_id>10010147.10010341.10010342.10010344</concept_id>
<concept_desc>Computing methodologies~Model verification and validation</concept_desc>
<concept_significance>500</concept_significance>
</concept>
<concept>
<concept_id>10011007.10011074.10011099.10011692</concept_id>
<concept_desc>Software and its engineering~Formal software verification</concept_desc>
<concept_significance>300</concept_significance>
</concept>
</ccs2012>
\end{CCSXML}

\ccsdesc[500]{Computing methodologies~Model verification and validation}
\ccsdesc[300]{Software and its engineering~Formal software verification}

%
%  Use this command to print the description
%
%\printccsdesc

%\keywords{Reachability; Hybrid Systems; Falsification}

\newcommand{\RA}[1]{R^{#1}}
\newcommand{\Map}[1]{f_{#1}}

%%%%%%%%%%%%%%%%%%%%%%%%%%%%%%
\section{Introduction}
\label{sec:intro}

In \chapref{abs}, we analyzed systems by restricting ourselves to
black box semantics. This enabled us to introduce an approach
independent of the system's structure (in practice). By observing the
dynamics only locally and sparsely (due to a coarse abstraction), we
were able to efficiently find abstract counter-examples. However, to
concretize them, we were forced to `take a closer look' or refine the
abstraction. As noted previously, splitting the state-space evenly is
an expensive operation in higher dimensions and we would like to avoid
it. We address this by using learning techniques to model the given
black box system by a PWA relational model.

In this chapter, we propose an approach which uses regression to
quantitatively estimate the discovered relations (witnessed by
trajectory segments) by affine maps. These maps are summaries of
locally observed behaviors, which we incorporate into the sampled
reachability graph as edge annotations. The resulting graph can be
interpreted as a discrete transition system and model checked for
time bounded safety properties. If a violation is discovered, it is
expected to be `close' to a true violation of the system.

\section{Overview}

In this chapter, we further our exploration of trajectory segment
based methods; and propose another approach for searching safety
violations in dynamical systems. Unlike the CEGAR based approach in
\chapref{abs} which directly uses the sampled reachability graph, we
first model quantitatively the local behaviors (represented by edges).
Recall that the edges of the graph represent observed trajectory
segments, between the respective cells. From this, we can only
conclude that a cell is reachable but cannot comment on the
underlying dynamics. If however, we estimate the dynamics of the
witnessed trajectory segments between two cells, we can model the
local dynamics. As we use affine templates to achieve this, it can be
compared to trajectory \textit{linearization}.
% We differentiate it by noting that

The dynamics can be either estimated soundly or only approximated with
error estimates. For the former, we require the white box model of a
system. Using a reach-set estimation tool like
\flowstar~\cite{chen2013flow}, the reachable states of an abstract
state can be soundly computed. As this is not usually the case, in the
rest of the chapter we focus on the case of black box systems.
Recall that such systems are equipped with a simulation function
$\simulate$, but the internal details of the model are opaque to us.
We use simple regression to fit an affine map to the locally
observed behaviors and use cross-validation to estimate the errors.
This results in an interval affine map, approximating a reachable
hyper-rectangle in the state-space.

Observe that the directed reachability graph can be interpreted as a
discrete transitions system. However, an explicit model checker can
only find abstract counter-examples in it. The graph does not have
enough information to enable the direct search of concrete
counter-examples. We remedy this by incorporating the estimated local
dynamics by annotating the graph edges with them. The resulting
transition system is rich enough that an off-the-shelf bounded model
checker can find concrete violations of a safety property, thus doing
away with the expensive refinement process. We now discuss the
background required to present our ideas.

%%%%%%%%%%%%%%%%%%%%%%%%%%%%%%

%%%%%%%%%%%%%%%%%%%%%%%%%%%%%%
\subsection{Motivation}
\label{sec:mot}
\input{motivation.tex}
%%%%%%%%%%%%%%%%%%%%%%%%%%%%%%

%%%%%%%%%%%%%%%%%%%%%%%%%%%%%%
\section{Related Work}
\label{sec:rel}
\input{related.tex}
%%%%%%%%%%%%%%%%%%%%%%%%%%%%%%

%%%%%%%%%%%%%%%%%%%%%%%%%%%%%%
\section{Prelims}
\label{sec:prelims}

\subsection{Relational Abstraction}

For continuous dynamical systems, it is hard to directly reason about
reachability. Discretization is often employed to transform the
systems into a discrete transition system. One idea is to abstract
continuous relations by \textit{reachability invariants}. The
resulting relations can be either time independent or time dependent.
Depending on the property at hand, we can select the appropriate one.
The resulting abstractions (usually discrete transitions systems) can
be analyzed using model checkers. We briefly discuss the two types of
relational abstractions.

\mypara{Time-less Relational Abstraction}

Relational abstractions for hybrid systems were proposed in
\cite{Sankaranarayanan+Tiwari/2011/Relational}. Proposed for hybrid
automaton models, the idea is to summarize the continuous dynamics of
each mode using a relation over reachable states. The resulting
relation is time independent and hence valid for all time as long as
the mode invariant is satisfied. The relations take the general form
of $R(\x,\x') \bowtie 0$, where $\bowtie$ represents one of the
relational operators $=, \ge, \le, <, >$. For \eg, an abstraction
which captures the monotonicity with respect to time for the
differential equation $\dot{x} = 2$ is $x' > x$.  The abstraction
capturing the relation between the set of ODEs: $\dot{x} = 2$,
$\dot{y} = 5$, is $5(x' - x) = 2(y' - y)$.

Such relations summarize the given system as a transition system.
These can be verified using $k$-induction or falsified using bounded
model checkers. The relationalization procedure involves finding
suitable abstractions, such as affine abstractions, eigen abstractions
and box abstractions~\cite{Sankaranarayanan+Tiwari/2011/Relational}.
For this, differnt techniques like template based invariant
generation~\cite{Gulwani+Tiwari/2008/Constraint,
Colon+Sankaranarayanan+Sipma/03/Linear} can be used.

\mypara{Timed Relational Abstraction}

As the above discussed relations are timeless, we cannot check their
timing properties.  Moreover, they cannot be easily specialized for
time-triggered systems. Timed relational abstractions provide a
similar solution by computing relations, but, time dependent ones.

For the case of dynamics defined by affine set of ODEs, timed
relations are provided by their solutions $\x(t) = e^{tA}\x(0)$.
This gives us the timed relation $\x' = e^{tA}\x$. Clearly, the
relation is non-linear with respect to time. However, for the case of
SDCS with a fixed time period, we obtain linear relations in $\x$.
We demonstrated the usefulness of timed relational abstractions on
linear systems in~\cite{zutshi2012timed}, and now incorporate the idea
to discretize black box dynamical systems.

%TODO
% \subsection{Time-Aware}
% Captures the behavior between two time instants which timed relational
% abstraction ignores.

\subsection{Simple Linear Regression}

\mypara{Regression}

In statistics, regression is the problem of finding a
\textit{predictor}, which can suitably predict the relationship
between the given set of observed input $\x$ and output $\y$ vectors.
In other words, assuming that $\y$ depends on $\x$, regression
strategies find either a parameterized or a non-parameterized
prediction function to explain the dependence. We now discuss simple
linear regression, which is parametric in nature and searches for an
affine predictor. It is also called \textit{ordinary least squares}.

\mypara{Ordinary Least Squares (OLS)}

Let the data set be comprised of $N$ input and output pairs $(\x,
y)$, where $\x\in\reals^n$ and $y\in\reals$. If $N>n$, which is
the case in the current context of \textit{finding the best fit}, the
problem is overdetermined; there are more equations than
unknowns. Hence, a single affine function cannot be found which
satisfies the equation $\forall i\in\{1\ldots N\}. y_i = A\x_i + \vb$. Instead, we
need to find the `best' choice for $A$ and $\vb$. This is formally
defined using a loss function. For the case of simple linear
regression or OLS, the loss function is the sum of squares of the
errors in prediction.  The task is then to determine the matrix of
coefficients $A$ and an offset vector $\vb$, such that the least
square error of the affine predictor is minimized for the given data
set.
\[
    \min_{A, b}\displaystyle\sum_{i=1}^{N}{\left(\y_i - (A\x_i + \vb)\right)^2}
\]
The solution of OLS can be analytically computed as
\[ A = (X^TX)^{-1}X^T\y\]
where $X$ is the matrix representing the horizontal stacking of all $\x$.
The details can be found in several texts on learning and statistics
\cite{friedman2001elements}.

\subsection{Piecewise Affine Model (PWA)}

A PWA model is a collection of guarded affine maps which compute the
next state of the underlying dynamical system. The affine maps
discretize the underlying continuous-time dynamics.

Given a state vector $\x\in\reals^n$, a guarded affine map
$\gm:(\guard, \amap)$ defines the discretized consecution rule as a
pair of an affine guard predicate $\guard:C\x - \vd \le 0$ and an affine map
$\amap:A\x+\vb$, where $A \in \reals^{n \times n}$, $C \in
\reals^{m \times n}$ are matrices and $\vb \in \reals^n$, $\vd
\in \reals^m$ are vectors. A guarded affine map is satisfied if its
guard is satisfied.

We now formalize the PWA affine model for a dynamical system.
\begin{definition}[PWA Model]

Given a dynamical system over a continuous state-space
$\ContStates\in\reals^n$, a PWA model is a map $\ContStates \mapsto
\setof{\gm_1, \gm_2, \ldots, \gm_n}$ from the state-space of the
dynamical system to a finite set of guarded affine maps
$\setof{\gm_1, \gm_2, \ldots, \gm_n}$. It defines the consecution
rule by the affine map of the satisfied $\gm$.

\begin{equation}
    \pwa = \left\{
        \begin{array}{ll}
            \gm_1: \amap_1(\x) &\text{if}\; \guard_1(\x) \\
            \gm_2: \amap_2(\x) &\text{if}\; \guard_2(\x) \\
            \ldots & \ldots\\
            \gm_n: \amap_n(\x) &\text{if}\; \guard_n(\x) \\
        \end{array}
    \right.
\end{equation}

\end{definition}

where $\amap_i = A_i\x + \vb_i$, and $\guard_i(\x) \equiv C_i\x-\vd_i\le0$.
Abusing the notation, we denote the next state computed by the PWA
model as $\x' = \pwa(\x)$. A PWA model is \textit{deterministic}, iff
for every state $\x\in\ContStates$, a unique guarded affine map is
satisfied. A PWA model is \textit{complete}, iff for every state
$\x\in\ContStates$, there exists at least one satisfied guarded affine
map $\gm$.
% $\forall x \in \ContStates. \forall \gm_i. $

%%%%%%%%%%%%%%%%%%%%%%%%%%%%%%

%%%%%%%%%%%%%%%%%%%%%%%%%%%%%%
\section{SCAMR}
\label{sec:scamr}
The plant abstraction developed in our earlier
work~\cite{zutshi2014multiple} was shown to be quite powerful in
finding falsifications when only a black box description was
available. The abstraction was built and explored using simulations
and unlike white-box approaches did not exploit the underlying
structure of the system $\scrS$. In this chapter we try to remedy this
by estimating a model of the system and recovering some structure of
the underlying system. Taking such a route can potentially improve the
previously used expensive refinement procedure and also provides us an
alternative perspective on the problem.

\section{Approach 1: Relation based refinement}
In the first approach, we propose to modify the refinement procedure
used by SCAM. Before we proceed, let us briefly recall the general
SCAM algorithm.

\begin{enumerate}
    \item Assume an $\epsilon$-grid.
    \item Scatter and Simulate.\label{s2}
    \item Build a Graph.
    \item Find abstract counterexamples (paths from the initial cell
        to the error cell).
    \item Concretize abstract counterexamples using simulations.
    \item Upon failure, refine by reducing $\epsilon$ and
        repeating from step~\ref{s2}.
\end{enumerate}

SCAM uses the abstract counterexamples to indicate the cells which
need to be refined. These cells are split into their constituent cells
which are then further refined till a concrete counterexample is
found.

The abstraction uses the below two rules to guide the search for
counterexamples.
\begin{itemize}
    \item Reachability.\\
        $\exists \vx_i \in C_i.\; \exists \vx'_i \in C'_i.\;
        \vx_i \areach{} \vx'_i \iff C_i \reach{} C'_i$

    \item Transitivity.\\
        $C_i \reach{} C'_i \wedge C'_i \reach{} C''_i \implies C_i \reach{} C''_i$
\end{itemize}
Given an abstract path $C_i\reach{}C'_i\reach{}C''_i$, the
refinement involves the search for a $\hat{\vx}_i \in C''_i$ and a
$\hat{\vx}''_i \in C_i$, such that $\hat{\vx}_i \areach{} \hat{\vx}$.

\begin{align*}
    \exists \vx_i \in C_i.\; \exists \vx'_i \in C'_i.\; \exists \vx''_i
    \in C'_i.\;  &\vx_i \areach{} \vx'_i \wedge \vx'_i \areach{} \vx''_i\\
    \implies &\exists \hat{\vx}_i \in C_i.\; \exists \hat{\vx}''_i
    \in C''_i.\; \hat{\vx}_i \areach{} \hat{\vx}''_i
\end{align*}
Although the above abstraction indicates the cells which should be
refined, it does not provide a good refinement procedure, and in fact
SCAM uses a brute force technique based on sampling based state search.

Instead, we wamt to find a local model for the evolution of the system
`along the abstract counterexample'. A model which is (a) cheap to
construct, (b) accurate locally, (c) amenable to search for a cocnrete
path. To achieve this we propose modeling of the abstract
counterexample using picewise affine relations. This can be either
done soundly using overapproximate set reachable techniques or by
using statistical regression.

Given an abstract relation $\reach{\tau}\subseteq(\C \times \C)$, we
define an enriched relation $\reach{\Map{}, \tau}$ over the concrete
states of the respective cell. In addition to stating reachability,
$\reach{\Map{}, \tau}$ also provides a local model for computing
concrete pairs of reachable states.

Hence, if $C_i \reach{} C_j$, then we enrich it by computing a
$\Map{i,j}$, such that $\forall \vx_i \in C_i.\; \exists \vx_j \in
C_j.\; \vx_i \areach{} \vx_j \implies \Map{i,j}(\vx_i) \approx \vx_j$.
If we can overapproximate $\Map{i,j}$, then we can bound the error
between $\Map{i,j}(\vx_i)$ and $\vx_j$. Selecting an affine template
has a key benefit; it enables efficient reasoning w.r.t. reachability.
Given a path in the abstraction, we can cast the problem of finding a
path in the enriched abstraction as the feasbility checking problem
for a set of linear inequalities.  A feasible solution to the
resulting linear program predicts a concrete path, the accuracy of
which depends on the local model.

\begin{align}
\min \vx^T \vc\;\; s.t.\\
\bigwedge_{i=0\ldots n} \vx_i \in C_i\\
\bigwedge_{i=0\ldots n} \vx_{i+1} \in A_{i,i+1}\vx_i + \delta
\end{align}

In summary, the new approach is equivalent to first finding the
reachability graph using SCAM, enumerating counterexamples (paths from
the initial cell to the error cell) and then annotating the associated
edges with affine relations. Each counterexample then represents a
linear program, with its feasible region predicting the concrete
counterexample.

If a path is infeasible, we can further analyse the infeasible
constraint set and perhaps comment about it. We can derive
qualitative as well as quantitative information, and use it to either
quantify how 'unreachable' a state is and why is not reachable. We
then discard the path.

%% This is not aaplicable in case  of sound models (either
% statistically sound against the simulations or sound against the
% the given white box model.
% improve our local models (which one and how) - Make sure they are
% overapproximae? and not missing a possible behavior.

%% Salient points: When can the PWA model miss a behavior described by
%% the originial white/black-box model?
%   - when the PWA model is not over aproximate
%   - due to the PWA model capturing behaviors only at discrete times.


Even when a path is feasible, we need to use our previously defined
concretization step to verify if the found initial state leads to an
error state. To recall, this is done by simply using the $\simulate$
function and simulating $k$-times at and around the $e$-neighbourhood
of the feasible region, where $k$ and $e$ define the budget of the
concretization step. If the concretization step fails, we can either
increase the budget or abandon the search, or refine our local models.
How do we refine local models? The error along the enriched path can
be computed. % by linearly transofrming the errors along each
% edge.

% The error at each each $x'$ can be computed. If error ball for the
% $x_f$ is contained entirely inside the unsafe set, then we know that
% we have a violation. If not, then its not clear which model to refine
% and how.
% - Sensitivity of the final path towards the constituent paths.
% - Splitting cells based on regression till the error is contained
% within a given threshold (fixed upfront). The subsequent iterations of
% SCAMR can work towards reducing this error.

% The local models can be refined iteratively by adding the new
% failure witnesses from the concretization step.

\section{Approach 2: PWA modeling and Falsification}

symbolic maps describing evolution of the system for each cell. This
is akin to computing a piecewise affine, discrete time model of the
system. It also provides us with a way to define relations between two
abstract states. We associate an an affine map $\Map{i}$ with every
abstract state $C_i$, which maps a concrete state in cell $C_i$ to
another concrete state reachable in one time step $\tau$. The time
step parameterized abstract relation $\RA{\tau}$ can be stated in
terms of these maps. Two abstract states are related, if the system
can evolve from one to another in time $\tau$.

$\Map{}$ can be either be soundly approximated using a regression
technique and numerical simulations or can be over-approximated by
symbolic reach set computation tools like flow*~\cite{chen2013flow}.
The resulting abstraction $\hat{\scrS}$ is non-deterministic, and can
be interpreted as a discrete time, piecewise affine model of the
underlying plant. A time bounded reachability question for the system
$\scrS$ can then be encoded as a BMC formula over finite steps and
checked over the abstraction $\hat{\scrS}$ using an SMT solver like
Z3~\cite{DeMoura+Bjorner/08/Z3}. If satisfiable, Z3 returns a path in
the abstraction which is a sequence of abstract states, and indicates
a potential concrete violating trajectory of the given system $\scrS$.
A concretization step is used to discover such a concrete violation.
On the other hand, an unsatisfiable formula indicates the absence of a
counterexample in the abstraction.

We first introduce the template of the piecewise affine model, present
a method to estimate it from a given black box system, and then detail
the resulting abstraction and propose an iterative falsification
methodology.

\paragraph {Assumptions.} We assume that the plant is described as a
black box system equipped with a forward numerical simulation function
$\simulate$.

\section{Answering the Reachability Question}

Approximating a continuous plant $\scrS$ by the abstract transition
system $\hat{\scrS}$ allows us to encode the reachability problem by a
BMC formula. As we kept the map $\Map{}$ affine, we can use
off-the-shelf SMT solvers like Z3~\cite{DeMoura+Bjorner/08/Z3} to
check it. We now formally state the problem.

\paragraph{Problem Statement.} Given a polytope representing unsafe of
states, $X_f: A_f\vx \le \vb_f$, a discrete transition system
$\hat{\scrS}$ defined by $\Map{}$, it's initial set of states (also a
polytope) $X_0: A_0\vx \le \vb_0$, a finite number of steps $N$, does
there exist a trajectory of the system from an initial state $\vx_0
\in X_0$ to an unsafe set such that $\vx_k \in X_f$, where $\vx_k =
\Map{}^k(\vx)$ and $k \le N$.

\subsection{Search for a violation.} Given a BMC formula over affine
arithmetic, an SMT solver either returns a satisfying assignment for
it or finds it unsatisfiable. The former case provide us with an
abstract violation while the latter conveys us the absence of one in
the sound abstraction. To find the corresponding concrete violation if
it exists, a concretization step is used.

\subsection{Concretization} An abstract path can be concretized by
using random simulations. Same as before, we use the initial abstract
state $C_0$ in the returned abstract path and sample it to get
$\scr{N}$ concrete states ($\scr{N}$ is the concretization budget).
The states are then simulated for specified time horizon $T$ using the
given simulation function $\simulate$. If a violation is triggered, we
conclude a successful falsification and stop. If not, we refine the
abstraction.

\subsection{Refinement} A refinement of the abstraction can be carried
out by reducing the error in the relation. The error is captured by
the interval $[\underline{\vb}, \overline{\vb}]$ in the map $\Map{}$.
Multiple strategies can be used to refine the maps, depending upon the
underlying method used to over-approximate the $\Map{}$. One way would
be to reduce the size of the cells by refining the $\epsilon$-tiling.
The interval affine maps thus found using the same method would be
more precise.

\section{Discussion}

There are a few known points which need to be addressed in this
approach. Particularly, an algorithm needs to be designed to explore
the cells and estimate their associated maps. An exhaustive
enumeration is undesirable, and hence there needs to be a mechanism to
prioritize which cells to explore. A potential candidate would be to
use the previously introduced scatter-and simulate algorithm.

Another aspect we need to discuss is the refinement procedure.
Assuming a finer $\epsilon$-tiling has been mentioned, but but perhaps
we can refine each cell individually as per their respective observed
over-approximation. Also, it needs to be investigated if a less
expensive and more flexible state-space decomposition can be used
instead of $\epsilon$-tiling, for \emph{e.g.} simplicial
decomposition.

Finally, if possible, we would like the time step $\tau$ to be dynamic
and dynamics based instead of fixing one for the entire system.


% TODO: Problems to be addressed...
% - the discrete system must not jump over an unsafe set.
% - pick a suitable time step $\tau$.
% - use a suitable algorithm (scatter and simulate) to explore the
% implicit abstraction.
% - collect the explored non-terminal cells in a list $CL$.
% - For each cell $C \in CL$, find a linear timed relation
% $\forall x \in C. R_{C}^{\tau}(x,x')$, which describes the
%  given system's reachable state $x'$ at time $t+\tau$ originating from
%  $x \in C$ at time $t$. For \emph{e.g.}, least squares can be used to
%  estimate such a linear map within a given tolerance.
% - The linear timed relations will be of the form $x' = Ax + b + Gu$,
% where $u$ is the input, and $A, b, G$ need to determined.
% - Once linear timed relations are found, we can do either of the
% following:
%     -- encode the cell behaviors as a transition system. Such a system
%     will take the form: $x \in C_i \implies R_{C_i}^{\tau}(x,x')$ A
%     concrete violation of bounded path length can now be searched by
%     encoding the reachability problem as a BMC  query and using an SMT
%     solver.
%     -- using the explored cells, construct a graph with the cells as
%     the nodes and the edges as relations between cells. The edges will
%     be annotated by the discovered linear timed relations. An abstract
%     violation query can now be formulated as a search for a path from
%     the initial cells to the unsafe cells. Such paths can then sent
%     off to an LP solver to find the existence of a concrete violation.

% - Assuming we found a concrete violation in the previous step, we can
% now proceed to concretize the violation for the actual system. If we
% are unable to, we can refine.
%
% - On the other hand, if we could not find any concrete violation in
% the estimated model, we again refine.



%%%%%%%%%%%%%%%%%%%%%%% discard....

% We define an abstraction relation $\RA{\tau}$, parameterized by the
% time step $\tau$ for each abstract state $C \in \scr{C}$, relating the
% concrete state pairs $(x_i,x'_i)$ under the input $u_i \in \Inputs$ as
% $\RA{\tau}(x_i,u_i,x'_i)$. We model this relation by the below map
% $f_C$.
%
% \[RA{\tau}(\vx_i,\vu_i,\vx'_i) \iff
    % \exists \vx_i \in C_i.\; \vx'_i = A\vx + \vb\]

%%%%%%%%%%%%%%%%%%%%%%%%%%%%%%

%%%%%%%%%%%%%%%%%%%%%%%%%%%%%%
\section{Implementation}
\label{sec:impl}

\section{Implementation and Evaluation}

The implementation was prototyped as S3CAM-R, an extension to our
previously mentioned tool S3CAM (\chapref{case}). OLS regression
routines were used from Scikit-learn~\cite{pedregosa2011scikit},
Python module for machine learning. SAL~\cite{SAL-SRI}
with Yices2~\cite{dutertre2014yices} was the model checker.

We tabulate our preliminary evaluation in \tabref{res-rel}. We use the
Van der Pol oscillator and the Brusselator, as described in the previous
chapter. As before, we ran S3CAM-R $10$ times with different seeds and
averaged the results. We tabulate both the total time taken and the
time taken by SAL to compute the counter-example and compare against
S3CAM.

\begin{table*}[!htbp]
\centering
\caption{Avg. timings for benchmarks. The \textbf{BMC} column lists time
    taken by the BMC engine. The total time is noted under
\textbf{S3CAM-R} and \textbf{S3CAM}.}
\label{tab:res-rel}
\begin{tabular}{@{}llll@{}}
\toprule
Benchmark & BMC & S3CAM-R & S3CAM\\
\midrule
Van der Pol ($\scr{P}3$)    & $0.0s$  & $10.0s$ & $15.0s$\\
Brusselator               & $0.2s$ & $4.0s$  & $2.5s$\\
%Lorenz                    & $0.$ & $s$  & $.s$\\
\bottomrule
\end{tabular}
\end{table*}

% 1m8s, 2m56.184s, 1m31.059s[f], 1m32.665s,  1m45.312s, 1m22.397s, 2m1.451s[f], 1m12.118s, 1m26.596s
% 0.12, 0.06,         0.33       0.13,       0.22,     FAIL,       ,X           0.08        0.15

The results are favorable and show promise. However, we need to
explore more benchmarks to conclude conclusively.

%%%%%%%%%%%%%%%%%%%%%%%%%%%%%%

%%%%%%%%%%%%%%%%%%%%%%%%%%%%%%
\section{Experimental Results}
\label{sec:res}
\input{results.tex}
%%%%%%%%%%%%%%%%%%%%%%%%%%%%%%

%%%%%%%%%%%%%%%%%%%%%%%%%%%%%%
\section{Conclusions}
\label{sec:concl}


\section{Summary}% and Future Work}

We have presented another methodology to find falsifications in black
box dynamical systems. Combining the ideas from abstraction based
search (\chapref{abs}), with our previous relational
abstractions~\cite{zutshi2012timed}, we created enriched abstractions.
These can be checked for safety violations using existing bounded
model checkers.  We used learning techniques to estimate the local
dynamics underlying trajectory segments, and approximated a transition
system from a black box system. Finally, we showed our approach on a
few examples.

As a future extension, we are working on a specialized BMC for the
problem at hand. We intend to explore efficient ways for encoding the
PWA relational system.

%  There are several directions we can take from here.

%  \paragraph{Impr}The biggest impediment to our approach are SMT solvers which use the
%  theory of reals with exact precision. This is important for
%  verification approaches, but not can be relaxed for falsification. An
%  SMT solver which uses approximate reasoning but returns robust
%  counter-examples will be very useful, though it doesn't exist as yet.
%  
%  Specializing BMC towards discretization
%  
%  \subsection{Challenges and Extensions}
%  Let us outline the challenges we face in this approach. Although
%  SAT/SMT solvers are very efficient owing to extensive engineering
%  effort, reachability of transition systems resulting from dynamical
%  systems remains a difficult problem.
%  
%  %Why exactly?
%  As the time horizon of the safety property increases, the possible
%  combinations of discrete transitions increases exponentially. Hence,
%  to find a counterexample which is a sequence of discrete transitions
%  over a `long' time horizon is not tractable for most but the simplest
%  of dynamical systems. % Solve using proper discretizations

%\section{Learning Discrete Abstraction for Black Box Systems}
%\section{Comparison with traditional Trajectory Linearization}
%\subsection{Piece-Wise Affine Model}
%\subsection{Other models}

%%%%%%%%%%%%%%%%%%%%%%%%%%%%%%

\bibliographystyle{abbrv}
\bibliography{references}


\end{document}
