\section{The Piecewise Affine Model}

The plant can be approximated by a switched discrete dynamical system
$\hat{\scrS}$ (or a transition system) by discretizing the dynamics in
space and time. The space is discretized by quantizing the continuous
states into discrete states and the time by fixing a time step $\tau$.

The dynamics are then defined for each discrete state by the maps
$\Map{i}: ContStates \mapsto ContStates$, with the overall evolution
given by

\begin{equation}
    \Map{} = \left\{
        \begin{array}{ll}
            \Map{1} = A_1\vx + \vb_1 & \vx \in C_1\\
            \ldots & \ldots\\
            \Map{i} = A_i\vx + \vb_i & \vx \in C_i\\
            \ldots & \ldots\\
            \Map{n} = A_n\vx + \vb_n & \vx \in C_n
        \end{array}
    \right.
\end{equation}

Given a current state of the system $\vx$, using the discretization we
can approximate the the state after time $\tau$ with some error
$\delta$ as follows
\begin{equation}
    \Map{i}(\vx) \approx \simulate(\vx, \tau),\;\; \vx \in C_i
\end{equation}

The discretization $\hat{\scrS}$ is qualified as $\hat{\scrS}_\delta$
if it is $\delta$-approximate, \ie, the below is true under a given
norm.
\begin{equation}
    \forall \vx \in ContStates.
        ||\Map{}(\vx) - \simulate(\vx, \tau)|| \le \delta
\end{equation}

$\hat{\scrS}$ is completely specified if the piecewise affine map
$\Map{}$ includes a $\Map{i}$ for each $C_i \in ContStates$.

The approximated discrete system $\hat{\scrS}$ evolves by the
application of the computed map $\Map{}$. Given a state $\vx$ of the system
at time $t$, the state at time $t + k\tau$, $\vx_k$ is approximated by
iterating $\Map{}$ over $\vx$, $k$ times.

\begin{align}
    \vx_k &= \Map{}(\vx_{k-1}) \nonumber\\
          &= \Map{}^2(\vx_{k-2}) \nonumber\\
          &= \ldots \nonumber\\
          &= \Map{}^k(\vx)
\end{align}

For a more rigorous analyses, we can bound error for a computed state
using the map $\Map{}$ as $||\Map{}(\vx_k) - \simulate(\vx, k\tau)||
\le \epsilon$, where $\epsilon$ can be computed for the given path
(given sequence of transformations).
%This enables us to compute a
%sound over-approximation of $\Map{}$ by generalizing $\vb$ to a vector
%of intervals. Abusing the notation, we keep the notation of the
%over-approximate map the same as of the approximate map. Henceforth
%$\Map{}$ denotes an over-approximation. We discuss this further in the
%next section.

Why do we want PWA models?
- Easy to construct
- Can be made arbitrarily precise
- Easy to analyse

\section{Abstraction}

We construct the abstraction of the given black box system $\scrS$ by
combining $\epsilon$-tiling based abstraction from the previous
chapter with the idea of timed relations from the first chapter. For
every cell $C_i$ (abstract state) induced by an $\epsilon$-tiling, we
associate an interval affine map $\Map{i}$ over-approximating the
evolution of the set of concrete states $\setof{\vx | \vx \in C_i}$
for a time step $\tau$. The abstraction is implicitly assumed and the
cells need not be enumerated. The maps can be computed on demand.

\paragraph{Symbolic Abstraction.} We define an abstraction relation
$\RA{\tau}$, parameterized by the time step $\tau$ for each abstract
state $C_i \in \scr{C}$ as $\RA{\tau}(C_i,C_j)$. The relation uses
$\Map{i}$ to over-approximate the states reachable from a concrete
state $\vx_i \in C_i$ in time $\tau$ by the set $\Map{i}(\vx_i)$.

%\begin{definition}[Abstract Relation]
    Two abstract states in $C_i$ and $C_j$ are related by
    \[\RA{\tau}(C_i,C_j) \iff \exists \vx_i \in C_i.\; \Map{i}(\vx_i)
    \in C_j\]
%\end{definition}

The map $\Map{i}$ for every abstract state $C_i$ can be approximated
soundly against a fixed number of observations using the provided
$\simulate$ function. It's general template is $A\vx + \vb \pm e$,
which can be written as $A\vx + [\underline{\vb}, \overline{\vb}]$
where $[\underline{\vb}, \overline{\vb}]$ denotes a vector of
intervals (element-wise) upper bounded by $\overline{\vb}$ and
lower bounded by $\underline{\vb}$.
%The parameters $A$ and $\vb$ can be determined using linear
%regression for black-box systems.

\paragraph{Remark.} If the system description is given as a white box,
for \emph{e.g.}, specified as a hybrid automata model, we can use a
symbolic reach set computation tool like flow* to over-approximate
$\Map{}$.


\section{Generating a PWA Model}
\label{sec:gen-pwa}
To generate such a model,
we need to define a (a) partition of the state space of the system and
(b) summarize the affine overapproximate dynamics for each partition.
Both (a) and (b) are inter dependant procedures. We now suggest two
different techniques to build piecewise affine abstractions.
\begin{enumerate}
    \item Non-Linear Symbolic reachability methods.~\label{SRM}
    \item Statistical Learning methods.~\label{SLM}
\end{enumerate}

Symbolic reachability techniques implemented in flow* can be utilized
to find the reachable states from a given set of states in time $t$,
along with an interval affine relation between them. Though more
expensive than simulations, we do get soundness guarentees against the
model.

If however, a model is not provided and the system is described as a
black box model, a model can be constructed using several learning
techniques~\cite{alur2014precise}.

using the provided $\simulate$ function and classifying the
data
