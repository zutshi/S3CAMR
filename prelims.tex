
\subsection{Relational Abstraction}

For continuous dynamical systems, it is hard to directly reason about
reachability. Discretization is often employed to transform the
systems into a discrete transition system. One idea is to abstract
continuous relations by \textit{reachability invariants}. The
resulting relations can be either time independent or time dependent.
Depending on the property at hand, we can select the appropriate one.
The resulting abstractions (usually discrete transitions systems) can
be analyzed using model checkers. We briefly discuss the two types of
relational abstractions.

\mypara{Time-less Relational Abstraction}

Relational abstractions for hybrid systems were proposed in
\cite{Sankaranarayanan+Tiwari/2011/Relational}. Proposed for hybrid
automaton models, the idea is to summarize the continuous dynamics of
each mode using a relation over reachable states. The resulting
relation is time independent and hence valid for all time as long as
the mode invariant is satisfied. The relations take the general form
of $R(\x,\x') \bowtie 0$, where $\bowtie$ represents one of the
relational operators $=, \ge, \le, <, >$. For \eg, an abstraction
which captures the monotonicity with respect to time for the
differential equation $\dot{x} = 2$ is $x' > x$.  The abstraction
capturing the relation between the set of ODEs: $\dot{x} = 2$,
$\dot{y} = 5$, is $5(x' - x) = 2(y' - y)$.

Such relations summarize the given system as a transition system.
These can be verified using $k$-induction or falsified using bounded
model checkers. The relationalization procedure involves finding
suitable abstractions, such as affine abstractions, eigen abstractions
and box abstractions~\cite{Sankaranarayanan+Tiwari/2011/Relational}.
For this, differnt techniques like template based invariant
generation~\cite{Gulwani+Tiwari/2008/Constraint,
Colon+Sankaranarayanan+Sipma/03/Linear} can be used.

\mypara{Timed Relational Abstraction}

As the above discussed relations are timeless, we cannot check their
timing properties.  Moreover, they cannot be easily specialized for
time-triggered systems. Timed relational abstractions provide a
similar solution by computing relations, but, time dependent ones.

For the case of dynamics defined by affine set of ODEs, timed
relations are provided by their solutions $\x(t) = e^{tA}\x(0)$.
This gives us the timed relation $\x' = e^{tA}\x$. Clearly, the
relation is non-linear with respect to time. However, for the case of
SDCS with a fixed time period, we obtain linear relations in $\x$.
We demonstrated the usefulness of timed relational abstractions on
linear systems in~\cite{zutshi2012timed}, and now incorporate the idea
to discretize black box dynamical systems.

%TODO
% \subsection{Time-Aware}
% Captures the behavior between two time instants which timed relational
% abstraction ignores.

\subsection{Simple Linear Regression}

\mypara{Regression}

In statistics, regression is the problem of finding a
\textit{predictor}, which can suitably predict the relationship
between the given set of observed input $\x$ and output $\y$ vectors.
In other words, assuming that $\y$ depends on $\x$, regression
strategies find either a parameterized or a non-parameterized
prediction function to explain the dependence. We now discuss simple
linear regression, which is parametric in nature and searches for an
affine predictor. It is also called \textit{ordinary least squares}.

\mypara{Ordinary Least Squares (OLS)}

Let the data set be comprised of $N$ input and output pairs $(\x,
y)$, where $\x\in\reals^n$ and $y\in\reals$. If $N>n$, which is
the case in the current context of \textit{finding the best fit}, the
problem is overdetermined; there are more equations than
unknowns. Hence, a single affine function cannot be found which
satisfies the equation $\forall i\in\{1\ldots N\}. y_i = A\x_i + \vb$. Instead, we
need to find the `best' choice for $A$ and $\vb$. This is formally
defined using a loss function. For the case of simple linear
regression or OLS, the loss function is the sum of squares of the
errors in prediction.  The task is then to determine the matrix of
coefficients $A$ and an offset vector $\vb$, such that the least
square error of the affine predictor is minimized for the given data
set.
\[
    \min_{A, b}\displaystyle\sum_{i=1}^{N}{\left(\y_i - (A\x_i + \vb)\right)^2}
\]
The solution of OLS can be analytically computed as
\[ A = (X^TX)^{-1}X^T\y\]
where $X$ is the matrix representing the horizontal stacking of all $\x$.
The details can be found in several texts on learning and statistics
\cite{friedman2001elements}.

\subsection{Piecewise Affine Model (PWA)}

A PWA model is a collection of guarded affine maps which compute the
next state of the underlying dynamical system. The affine maps
discretize the underlying continuous-time dynamics.

Given a state vector $\x\in\reals^n$, a guarded affine map
$\gm:(\guard, \amap)$ defines the discretized consecution rule as a
pair of an affine guard predicate $\guard:C\x - \vd \le 0$ and an affine map
$\amap:A\x+\vb$, where $A \in \reals^{n \times n}$, $C \in
\reals^{m \times n}$ are matrices and $\vb \in \reals^n$, $\vd
\in \reals^m$ are vectors. A guarded affine map is satisfied if its
guard is satisfied.

We now formalize the PWA affine model for a dynamical system.
\begin{definition}[PWA Model]

Given a dynamical system over a continuous state-space
$\ContStates\in\reals^n$, a PWA model is a map $\ContStates \mapsto
\setof{\gm_1, \gm_2, \ldots, \gm_n}$ from the state-space of the
dynamical system to a finite set of guarded affine maps
$\setof{\gm_1, \gm_2, \ldots, \gm_n}$. It defines the consecution
rule by the affine map of the satisfied $\gm$.

\begin{equation}
    \pwa = \left\{
        \begin{array}{ll}
            \gm_1: \amap_1(\x) &\text{if}\; \guard_1(\x) \\
            \gm_2: \amap_2(\x) &\text{if}\; \guard_2(\x) \\
            \ldots & \ldots\\
            \gm_n: \amap_n(\x) &\text{if}\; \guard_n(\x) \\
        \end{array}
    \right.
\end{equation}

\end{definition}

where $\amap_i = A_i\x + \vb_i$, and $\guard_i(\x) \equiv C_i\x-\vd_i\le0$.
Abusing the notation, we denote the next state computed by the PWA
model as $\x' = \pwa(\x)$. A PWA model is \textit{deterministic}, iff
for every state $\x\in\ContStates$, a unique guarded affine map is
satisfied. A PWA model is \textit{complete}, iff for every state
$\x\in\ContStates$, there exists at least one satisfied guarded affine
map $\gm$.
% $\forall x \in \ContStates. \forall \gm_i. $
